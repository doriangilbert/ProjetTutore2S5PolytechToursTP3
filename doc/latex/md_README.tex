Il s\textquotesingle{}agit de réaliser l\textquotesingle{}analyseur syntaxique pour le langage J\+S\+ON

\subsection*{mode d\textquotesingle{}emploi}

\subsubsection*{1ere étape \+: récupérer le matériel initial \+:}

{\ttfamily git clone \href{https://gitlab.com/nicolas.monmarche/tp_il2_ex3.git}{\tt https\+://gitlab.\+com/nicolas.\+monmarche/tp\+\_\+il2\+\_\+ex3.\+git}}

\subsubsection*{2ieme étape créer votre branche (remplacer nom-\/binome par ce qui vous concerne) \+:}

{\ttfamily git branch \char`\"{}nom-\/binome\char`\"{}}

{\ttfamily git checkout \char`\"{}\+Nom-\/binome\char`\"{}}

\subsubsection*{3ieme étape transférer votre commit initial}

{\ttfamily git commit -\/m \char`\"{}notre premiere version\char`\"{}}

{\ttfamily git push -\/-\/set-\/upstream origin \char`\"{}\+Nom-\/binome\char`\"{}}

\subsubsection*{n-\/étape}

à chaque fois que vous souhaitez tranférer votre travail \+:

{\ttfamily git commit -\/m \char`\"{}explications\char`\"{} fichiers-\/modifiés}

{\ttfamily git push} 